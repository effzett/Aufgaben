
\documentclass[paper=a4,notitlepage,parskip=half,plainheadsepline,12pt]{scrartcl}
 
\usepackage{scrlayer-scrpage} 
% \usepackage{ngerman}
% \usepackage{graphicx}
% \usepackage{float}
% \usepackage[T1]{fontenc}
% \usepackage[utf8]{inputenc}
% \usepackage{pifont}
% \usepackage{amsmath}
% \usepackage[german]{varioref}
\usepackage[text={6.8in,9.6in},top=0.8in,centering]{geometry}
\usepackage{datetime2}
\usepackage{array,multirow,tabularx}
% \usepackage{listingsutf8}
\usepackage[table,svgnames,dvipsnames]{xcolor}
% \usepackage[most,breakable,skins,listingsutf8]{tcolorbox}
% %\tcbuselibrary{listingsutf8}
\usepackage[color=blue!90,scale=0.8,placement=bottom,vshift=1.2cm]{background}
%\usepackage{array}
% \usepackage{physics}
\usepackage{fancyvrb}
\usepackage{tikz}
\usetikzlibrary{calc}


%\backgroundsetup{contents={Version vom \today}}
\backgroundsetup{contents={Version vom \today}}
%\pagestyle{empty}

\pagestyle{scrheadings}
\clearscrheadfoot
\ihead{Fachinformatiker Anwendungsentwickler (1.~Ausbildungsjahr~2020)}
\chead{}
\ohead{Tipp: Aufgabe2}
\ifoot{Frank Zimmermann}
\cfoot{SVLFG}
\ofoot{}
\ofoot{\pagemark}

\graphicspath{{./jpg/}}
\DeclareGraphicsExtensions{.jpg,.eps,.png}
%\usepackage{courier}



% Wenn pdflatex benutzt wird sollen die 
% installierten! frutiger-Fonts verwendet werden
% Ansonsten die TeX-eigenen cmbright
\ifpdfoutput %
{%
\usepackage[scaled=0.90]{frutiger}
\renewcommand\familydefault{\sfdefault}
%\DeclareFixedFont\ott{T1}{phv}{mc}{n}{10pt
}%
{%
\usepackage{cmbright}
}%

\newfont{\lf}{svlfg scaled 2000}%
\VerbatimFootnotes
\begin{document}



\begin{center}
{\lf K}\hfill {\LARGE Tipp für Aufgabe 2} \hfill {\lf K}\\
\rule{\textwidth}{2pt}
\end{center}

In der Aufgabe~2 soll ein Buchstabe aus einer Verteilung von verschiedenen Buchstaben gewählt werden.
Dabei soll die Häufigkeit der vorkommenden Buchstaben berücksichtigt werden, aber ansonsten der ausgewählte Buchstabe zufällig gewählt werden.

Eine Lösung für diese Aufgabe, einerseits die Häufigkeiten zu berücksichtigen und andererseits aber einen Buchstaben zufällig auszuwählen, sei hier kurz skizziert.

Nehmen wir an wir hätten 3 verschiedene Buchstaben $b_1, b_2$ und $b_3$ mit den Häufikeiten $h_1, h_2$ und $h_3$.
In der Datenstruktur--Grafik wären das z.b. \fcolorbox{black}{blue!20}{m}, \fcolorbox{black}{red!20}{t} und \fcolorbox{black}{green!20}{s} mit den Häufigkeiten \fcolorbox{black}{blue!20}{12}, \fcolorbox{black}{red!20}{3} und \fcolorbox{black}{green!20}{7}. 

\begin{center}
\begin{tikzpicture}[scale=0.5]
\draw [ step =1 cm , gray , very thin ] ( 0 , 0) grid
(15 ,15) ;
\filldraw [fill=blue!20](0,0) rectangle (1,12);
\filldraw [fill=red!20](1,0) rectangle (2,3);
\filldraw [fill=green!20](2,0) rectangle (3,7);
\draw (0.5,6) node [rotate=90]{m};
\draw (1.5,1.5) node [rotate=90]{t};
\draw (2.5,3.5) node [rotate=90]{s};
\foreach \y/\ytext in {1,2,3,4,5,6,7,8,9,10,11,12} \draw (-5pt,\y) -- (-1pt,\y) node[anchor=east] {$\ytext$};
\end{tikzpicture}
\end{center}

Wenn man die Häufigkeiten als Intervalle betrachtet und nebeneinander anordnet, bekommt man ein 3--geteiltes Intervall von der Länge 22.

\begin{center}
\begin{tikzpicture}[scale=0.5]
\draw [ step =1 cm , gray , very thin ] ( 0 , 0) grid
(25 ,1) ;
\filldraw [fill=blue!20](0,0) rectangle (12,1);
\filldraw [fill=red!20](12,0) rectangle (15,1);
\filldraw [fill=green!20](15,0) rectangle (22,1);
\draw (6,0.5) node {m};
\draw (13.5,0.5) node {t};
\draw (18.5,0.5) node {s};
\draw[->,red,line width=2pt] (5,2) -- (13,2)   -- (13,1);
\draw (3.5,2) node [red]{random};
\draw [red, line width = 2pt](0,0) -- (0,2);
\draw [red, line width = 2pt](22,0) -- (22,2);
\foreach \x/\xtext in {1,2,3,4,5,6,7,8,9,10,11,12,13,14,15,16,17,18,19,20,21,22,23,24,25} \draw (\x ,5pt) -- (\x ,-25pt) node[anchor=north] {$\xtext$};
\end{tikzpicture}
\end{center}

Bestimmt man nun eine Zufallszahl im Bereich $0\dots 22$ und berechnet\footnote{in Pseudo-Code z.B.
\begin{verbatim} 
r=random(0..22);
i=0;
while(Intervallende[i]<r)(i++;);
wähle Buchstabe aus Intervall[i];
\end{verbatim}
}, in welches Intervall sie fällt, so hat man die Größe der Intervalle bei der Auswahl der Zufallszahl berücksichtigt. 
\end{document}
